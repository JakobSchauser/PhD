\documentclass[11pt,a4paper]{article}
\usepackage[utf8]{inputenc}
\usepackage[T1]{fontenc}
\usepackage{lmodern}
\usepackage{amsmath,amssymb}
\usepackage{graphicx}
\usepackage{siunitx}
\usepackage{geometry}
\geometry{margin=2.5cm}

\title{FYS2: Spektrallinjeanalyse omkring \SI{5087}{\angstrom}}
\author{}
\date{}

\begin{document}
\maketitle

\section*{Formål}
Formålet med denne øvelse er at lære at:
\begin{itemize}
  \item indlæse et spektrum med tilhørende usikkerheder,
  \item visualisere data med akse\-labels og en meningsfuld titel,
  \item bestemme en linjecentroid og linjebredde (FWHM) med to metoder:
  (i) modeltilpasning med \texttt{curve\_fit} og (ii) en parameter-fri momentmetode,
  \item estimere usikkerheder på de målte størrelser.
\end{itemize}

\section*{Data}
Datafilen indeholder tre kolonner med header:
\texttt{wavelength}, \texttt{calibrated\_flux}, \texttt{flux\_var}.
Usikkerheden fås som $\sigma=\sqrt{\texttt{flux\_var}}$.
Filnavn: \texttt{sdssJ133702+305539\_combined.dat}.

\section*{Opgaver}
\subsection*{1) Indlæsning og plot}
\begin{enumerate}
\item Indlæs filen (f.eks.\ med \texttt{numpy.genfromtxt(..., names=True)}).
\item Lav et plot af hele spektret med akse\-labels (x: Wavelength [\si{\angstrom}], y: Flux [arb. units]) og en titel.
\end{enumerate}

\subsection*{2) Modeltilpasning (curve\_fit)}
\begin{enumerate}
\item Ekstrahér et vindue omkring linjen ved \SI{5087}{\angstrom} (f.eks.\ $\pm\SI{12}{\angstrom}$).
\item Antag en model bestående af en Gaussisk linje på en lineær kontinuum: 
\[ f(\lambda) = A \exp\!\left[-\frac{1}{2}\left(\frac{\lambda-\mu}{\sigma}\right)^2\right] + m \lambda + b. \]
\item Brug \texttt{scipy.optimize.curve\_fit} med de givne usikkerheder som vægte (\texttt{sigma=...}, \texttt{absolute\_sigma=True}).
\item Bestem centroid $\mu$ og FWHM $= 2\sqrt{2\ln 2}\,\sigma$ samt deres usikkerheder fra kovariansmatricen.
\end{enumerate}

\subsection*{3) Parameter-fri metode (momenter)}
\begin{enumerate}
\item Estimér og subtrahér kontinuummet i vinduet (f.eks.\ lineær fit på vinduets kanter).
\item Beregn første moment (centroid) og andet centralmoment (varians) for den kontinuum\-subtraherede linje:
\[ \mu = \frac{\sum \lambda\, I}{\sum I}, \qquad
\sigma^2 = \frac{\sum (\lambda-\mu)^2 I}{\sum I}, \]
hvor $I$ er (ikke-negative) line-fluxværdier efter kontinuum-subtraktion.
\item Udled FWHM fra $\sigma$ som ovenfor.
\item Estimér usikkerheder ved bootstrap: træk tilfældig støj fra en normalfordeling med standardafvigelser givet af \texttt{flux\_var} og gentag momentberegningen mange gange (f.eks.\ 300--1000 gange). Brug middelværdi og standardafvigelse af de resulterende fordelinger for $\mu$ og FWHM som estimater.
\end{enumerate}

\subsection*{4) Rapportér resultater}
\begin{itemize}
\item Præsenter $\mu$ og FWHM for begge metoder inkl.\ usikkerheder.
\item Diskutér evt.\ systematiske forskelle (modelantagelser, kontinuumestimat, vinduesvalg, støj).
\item Vedlæg et figurpanel med (i) fuldt spektrum og (ii) zoom-vinduet med bedste fit indtegnet.
\end{itemize}

\section*{Hints og tjekpunkter}
\begin{itemize}
\item Startgæt til \texttt{curve\_fit}: maksimum\-position som $\mu$, amplituden som (maks--median), $\sigma$ som en brøkdel af vinduesbredden.
\item Kontroller at FWHM bliver positiv og at $\sigma$ ikke presses til numeriske grænser; brug evt.\ bounds.
\item Husk at bootstrap-resultater kan afhænge af vinduesvalg og kontinuum\-model.
\end{itemize}

\section*{Ekstra (valgfrit)}
\begin{itemize}
\item Udvid kontinuum\-modellen (f.eks.\ 2.~grad polynomium) og sammenlign.
\item Undersøg følsomheden af resultaterne for forskellige vinduesstørrelser.
\item Hvis linjen er i absorption, tilpas fortegnet i modellen eller arbejd på inverteret flux.
\end{itemize}

\end{document}
